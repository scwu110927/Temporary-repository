% !TEX TS-program = xelatex
% !TEX encoding = UTF-8 Unicode
%\documentclass[14pt, handout]{beamer}				% for print
\documentclass[11pt]{beamer}			% for projector
%- begin of Beamer setting------------------------------------------------------------------------------------------------------------------

\usetheme[secheader]{Boadilla}
\usefonttheme{structurebold}
\useinnertheme{circles}
%\usepackage{xmpmulti}
\linespread{1.3}
% 設定顏色 color Table: http://latexcolor.com
\definecolor{airforceblue}{rgb}{0.36, 0.54, 0.66}
\usecolortheme[named=airforceblue]{structure}


%- end of Beamer setting--------------------------------------------------------------------------------------------------------------------

\usepackage{fontspec}								% Font selection for XeLaTeX
\usepackage{xeCJK}									% 中文使用 XeCJK
\usepackage{xunicode}								% Unicode support 
\usepackage{amsmath, amssymb, bm}					% amsthm
%\usepackage{wasysym}								% 特殊符號 http://detexify.kirelabs.org/symbols.html
\usepackage{enumerate}
\usepackage{array}
\usepackage{graphicx, subfig, float}				% 圖片套件
\usepackage{booktabs, lscape, threeparttable, multirow}
\usepackage{pgf,pgfpages}
\usepackage{cancel}
%\usepackage{fourier}								% Adobe Utopia
\usepackage{listings}
\usepackage{hyperref}
\usepackage{natbib}									% 參考書目
\usepackage{multicol}								% 目錄兩欄

%-------------------------------------------------------------------------------------------------------------------------------------------
% 主字型設定
\setCJKmainfont{cwTeX Q Fangsong Medium}			% 設定中文內文字型
				[BoldFont=cwTeX Q Hei Bold]			% 設定中文粗體字型
\setmainfont{Times New Roman}						% 設定英文內文字型
\setsansfont{Cambria}								% 設定英文無襯袖字型 used with {\sffamily }
\setmonofont{Courier New}							% 設定英文等寬度字型 used with {\ttfamily }

% 其他字型設定			
\newfontfamily{\A}{Arial}
\newfontfamily{\SC}[Scale=0.9]{Cambria}
\newfontfamily{\SCN}[Scale=0.9]{Courier New}
\newfontfamily{\TT}[Scale=0.8]{Times New Roman}
%\newCJKfontfamily{\MB}{微軟正黑體}
%\newCJKfontfamily{\SM}[Scale=0.8]{新細明體}			% 縮小版新細明體
%\newCJKfontfamily{\K}{標楷體}
\newCJKfontfamily{\CF}{cwTeX Q Fangsong Medium}		% CwTex 仿宋體
\newCJKfontfamily{\CB}{cwTeX Q Hei Bold}			% CwTex 粗黑體
\newCJKfontfamily{\CK}{cwTeX Q Kai Medium}   		% CwTex 楷體
\newCJKfontfamily{\CM}{cwTeX Q Ming Medium}			% CwTex 明體
\newCJKfontfamily{\CY}{cwTeX Q Yuan Medium}			% CwTex 圓體	

%-------------------------------------------------------------------------------------------------------------------------------------------

\XeTeXlinebreaklocale "zh"                  		%這兩行一定要加,中文才能自動換行
\XeTeXlinebreakskip = 0pt plus 1pt     				%這兩行一定要加,中文才能自動換行

%-----------------------------------------------------------------------------------------------------------------------
\renewcommand{\tablename}{表}						% 改變表格標號文字為中文的「表」(預設為 Table)
\renewcommand{\figurename}{圖}						% 改變圖片標號文字為中文的「圖」(預設為 Figure)

\usebackgroundtemplate
{
%\includegraphics[width=\paperwidth, height=\paperheight]{background.jpg}
}

%--------------------------------------------------------------------------------------------------------------------------------------------
\newcommand{\fib}[1]{\left(#1\right)}		% 設定 ()
\newcommand{\seb}[1]{\left[#1\right]}		% 設定 []
\newcommand{\thb}[1]{\left\{#1\right\}}		% 設定 []
\newcommand{\bsq}{\hfill$\blacksquare$}    	% 設定 \blacksquare
\newcommand{\pr}{\text{P\,}}    			% 設定 P
\newcommand{\Exp}{\text{E\,}}    			% 設定 E
\newcommand{\var}{\text{Var\,}}    			% 設定 Var
\newcommand{\cov}{\text{Cov\,}}    			% 設定 Var




