% These two lines must be incldued to open file under UTF-8
% !TEX TS-program = xelatex								
% !TEX encoding = UTF-8
% 套件設定
%\documentclass[11pt, a4paper, fleqn, titlepage]
% 			  {article}			            % 開起是否首頁置中
\documentclass[11pt, a4paper, fleqn]
 			  {article} 						
\usepackage{fontspec} 						% Font selection for XeLaTeX 
\usepackage{fourier}						% Adobe Utopia
\usepackage{xeCJK}							% 中文使用 XeCJK.
\usepackage{xunicode} 						% Unicode support for LaTeX character names.
\usepackage{amsmath, amssymb, amsthm, bm}	% math­e­mat­i­cal fea­tures found in AMS-TEX.
\usepackage{nccmath}						% cnter equation.
\usepackage{cases}				% Lists.
\usepackage{graphicx, subfig, float} 		% Support the \includegraphics command.
\usepackage{array, booktabs, threeparttable}% An ex­tended tab­u­lar en­vi­ron­ments.
\usepackage{indentfirst}					% 首行縮排
\usepackage{enumerate}
\usepackage[shortlabels]{enumitem}
\setlist{nosep}
\parindent=2em
\usepackage{xcolor}
\usepackage{listings}						% 映出程式碼
\usepackage[left=1.5cm, right=1.5cm, 
			bottom=2cm, top=25mm, 
			marginparwidth=0pt, 
			headheight=25mm]{geometry}		% cus­tomize page lay­out.
%\usepackage[left=1.5in,right=1in,
%			top=1in,bottom=1in]{geometry} 
\usepackage{setspace}						% 指定行距
\usepackage{fancyhdr}						% 套用頁首頁尾
%\usepackage[bibstyle=numeric]{biblatex}		% 參考書目

%\usepackage[style=reading,
%			bibstyle=numeric]{biblatex}		% 參考書目(社會學)\parencite{}
%\usepackage{dcolumn}
%\usepackage{longtable}
%\usepackage{colortbl}
%\usepackage{wrapfig, subfig, sidecap}		% 圍繞圖片,並排圖片,圖名放置。
%\usepackage{multicol}						% 多欄
\usepackage{tasks}
\settasks{label=(\alph*) ,label-width=1.5em,
          after-item-skip=-0.5em
          }
%\usepackage{icomma}						% 在數學式中,逗點後空格。
%--------------------------------------------------------------------------------------------------------------------------------------------
% 主字型設定
\setCJKmainfont{cwTeX Q Ming Medium}		% 設定中文內文字型
				[BoldFont=cwTeX Q Hei Bold]	% 設定中文粗體字型
\setmainfont{Times New Roman}				% 設定英文內文字型
\setsansfont{Arial}							% 設定英文無襯袖字型 used with {\sffamily ...}
\setmonofont{Courier New}					% 設定英文等寬度字型 used with {\ttfamily ...}


%--------------------------------------------------------------------------------------------------------------------------------------------
% 其他字型設定
\newfontfamily{\C}{Cambria}				
\newfontfamily{\A}{Arial}
\newfontfamily{\SCN}[Scale=0.9]{Courier New}
\newfontfamily{\TT}[Scale=0.8]{Times New Roman}
%\newCJKfontfamily{\MB}{微軟正黑體}
%\newCJKfontfamily{\SM}[Scale=0.8]{新細明體}		% 縮小版新細明體
%\newCJKfontfamily{\K}{標楷體}
\newCJKfontfamily{\CF}{cwTeX Q Fangsong Medium}	% CwTex 仿宋體
\newCJKfontfamily{\CB}{cwTeX Q Hei Bold}		% CwTex 粗黑體
\newCJKfontfamily{\CK}{cwTeX Q Kai Medium}   	% CwTex 楷體
\newCJKfontfamily{\CM}{cwTeX Q Ming Medium}		% CwTex 明體
\newCJKfontfamily{\CY}{cwTeX Q Yuan Medium}		% CwTex 圓體

%--------------------------------------------------------------------------------------------------------------------------------------------
\XeTeXlinebreaklocale "zh"					%這兩行一定要加,中文才能自動換行
\XeTeXlinebreakskip = 0pt plus 1pt			%這兩行一定要加,中文才能自動換行

%--------------------------------------------------------------------------------------------------------------------------------------------
\newcommand{\fib}[1]{\left(#1\right)}		% 設定 ()
\newcommand{\seb}[1]{\left[#1\right]}		% 設定 []
\newcommand{\thb}[1]{\left\{#1\right\}}		% 設定 []
\newcommand{\bsq}{\hfill$\blacksquare$}    	% 設定 \blacksquare
\newcommand{\pr}{\text{P\,}}    				% 設定 P
\newcommand{\Exp}{\text{E\,}}    				% 設定 E
\newcommand{\var}{\text{Var\,}}    			% 設定 Var
\newcommand{\cov}{\text{Cov\,}}    			% 設定 Var
\newcommand{\Oimgdir}{../image/}			% 設定圖檔的位置
\newcommand{\imgdir}{Code/}			    % 設定圖檔的位置
\renewcommand{\tablename}{表}				% 改變表格標號文字為中文的「表」(預設為 Table)
\renewcommand{\figurename}{圖}				% 改變圖片標號文字為中文的「圖」(預設為 Figure)
%--------------------------------------------------------------------------------------------------------------------------------------------
% 設定顏色 color Table: http://latexcolor.com
\definecolor{slight}{gray}{0.9}						

%--------------------------------------------------------------------------------------------------------------------------------------------
% 映出程式碼 \begin{lstlisting}
\lstset
{	language=R,  %[LaTeX]TeX
    breaklines=true,
    basicstyle=\linespread{0.5}\SCN,
    keywordstyle=\color{black}\bfseries,
    identifierstyle=\color{black},
%   commentstyle=\color{limegreen}\itshape,
%   stringstyle=\sffamily,
    showstringspaces=false,
    frame=single,							%default frame=none 
    rulecolor=\color{black},
    framerule=0.2pt,						%expand outward 
    framesep=3pt,							%expand outward
    xleftmargin=0em,						%to make the frame fits in the text area. 
    xrightmargin=0em,						%to make the frame fits in the text area. 
    tabsize=4,								%default :8 only influence the lstlisting and lstinline.
    escapeinside=`'                         %to cancel listing in part
}
 
\pagestyle{fancy}
\fancyhead[L]{Department of Statistics, NCCU}
\fancyhead[R]{存活分析與測量誤差研究}
\fancyfoot[C]{\thepage}
\title{{\CF HW4}}
\author{{\CF 109354003\  統研所二\  吳書恆}}
\date{{\CF \today}}
%文章開始---------
\begin{document}

\maketitle \thispagestyle{fancy}
\fontsize{11}{20 pt}\selectfont

\begin{enumerate}
%------------------------------------------------------------------
\item Generate 500 samples of $(X, Y)$ with $X\sim \text{Ber}(0.5)$ and $Y\sim \text{Ber}(0.3*X+0.6*(1-X))$. Test $H_0: p_0=p_1$, where $p_x=P(Y=1|X=x)$.\\
According to the question, \[Y|X=0\sim\text{Ber}(0.6)\text{ and }Y|X=1\sim\text{Ber}(0.3).\] Using R to generate samples and compute Chi-squared test.
\begin{lstlisting}
> set.seed(1234)
> X <- rbinom(500, 1, 0.5)
> Y <- rbinom(500, 1, 0.3*X+0.6*(1-X))
> (XY.tab <- table(X, Y))
   Y
X     0   1
  0 100 143
  1 176  81
> chisq.test(XY.tab)

	Pearson's Chi-squared test with Yates' continuity correction

data:  XY.tab
X-squared = 36.629, df = 1, p-value = 1.429e-09
\end{lstlisting}
Since $\chi^2 = 36.629$ and p-value $<.001$, reject $H0$. That's acceptable, because the Y distribution is different depended on X.

%------------------------------------------------------------------
\item Generate $W$ with $P(W=1|X=1)=0.9$ and $P(W=1|X=0)=0.75$.
\begin{enumerate}[(a)]
\item Does $W|X=x$ have anything to do with $Y$?\\
According to the question, the probability of $W|X=x$ is given and it doesn't contain any information about $Y$. Moreover, the information about Y can be completely determined by $X$. Hence, $W|X=x$ is independent to $Y$.

\item Find $(\theta_{1|1}, \theta_{1|0})$.\\
Use simulation data $(W, X, Y)$ to be true data, the true values of $(\theta_{1|1}, \theta_{1|0})$ can be calculated as follow.
\begin{lstlisting}
> W <- rbinom(500, 1, 0.9*X+0.75*(1-X))
> (WXY.tab <- table(W, X, Y))
W, X, Y = 0
      0   1
  0  25  17
  1  75 159

W, X, Y = 1
      0   1
  0  29   9
  1 114  72
\end{lstlisting}
\[\theta_{1|1}=P(W=1|X=1)=(159+72)/(17+159+9+72)=0.899\]
\[\theta_{1|0}=P(W=1|X=0)=(75+114)/(25+75+29+114)=0.778\]
\item Find $\frac{n_{11}}{n_{.1}}$ and compare with $\alpha_1\theta_{1|1}+(1-\alpha_1)\theta_{1|0}$.
\[\frac{n_{11}}{n_{.1}}=(114+72)/(29+9+114+72)= 0.830\] And, 
\[\alpha_1=P(X=1|Y=1)=(9+72)/(29+9+114+72)=0.362\]
\[\alpha_1\theta_{1|1}+(1-\alpha_1)\theta_{1|0}=0.362\times 0.899+(1-0.362)\times 0.778 = 0.822\]
The values of $\frac{n_{11}}{n_{.1}}$ and $\alpha_1\theta_{1|1}+(1-\alpha_1)\theta_{1|0}$ are $0.830$ and $0.822$ respectively, which are close to each other.

\item Find $\frac{n_{11}}{n_{.1}}-\frac{n_{10}}{n_{.0}}$ and compare with $(\alpha_1-\alpha_0)(\theta_{1|1}+\theta_{0|0}-1)$.
\[\frac{n_{10}}{n_{.0}}=(75+159)/(25+17+75+159)= 0.848\]
\[\frac{n_{11}}{n_{.1}}-\frac{n_{10}}{n_{.0}}= 0.830-0.847= -0.017\] And,
\[\alpha_0=P(X=1|Y=0)=(17+159)/(25+17+75+159)=0.638\]
\[(\alpha_1-\alpha_0)(\theta_{1|1}+\theta_{0|0}-1)=(\alpha_1-\alpha_0)(\theta_{1|1}-\theta_{1|0})=-0.033\]
The values of $\frac{n_{11}}{n_{.1}}-\frac{n_{10}}{n_{.0}}$ and $(\alpha_1-\alpha_0)(\theta_{1|1}+\theta_{0|0}-1)$ are $-0.017$ and $-0.033$ respectively, which are close to each other either.

\end{enumerate}
%------------------------------------------------------------------
\item Generate 100 external validation set of $(X^v, W^v)$ with $X^v\sim \text{Ber}(0.4)$ , $P(W^v=1|X^v=1)=0.9$ and $P(W^v=1|X^v=0)=0.75$.
\begin{enumerate}[(a)]
\item Find $(\hat\theta_{1|1}, \hat\theta_{1|0})$ and $\hat\alpha_1-\hat\alpha_0$.\\
$(\hat\theta_{1|1}, \hat\theta_{1|0})$ is calculate by simulation as follow
\begin{lstlisting}
> Xv <- rbinom(100, 1, 0.4)
> Wv <- rbinom(100, 1, 0.9*Xv+0.75*(1-Xv))
> (XWv.tab <- table(Wv, Xv))
   Xv
Wv   0  1
  0 16  2
  1 51 31
> theta1 <- XWv.tab[2, 2]/sum(XWv.tab[, 2])
> theta0 <- XWv.tab[2, 1]/sum(XWv.tab[, 1])
> c(theta1, theta0)
[1] 0.9393939 0.7611940
\end{lstlisting}
\[\hat\theta_{1|1}=31/(2+31)= 0.939\]
\[\hat\theta_{1|0}=51/(16+51)= 0.761 = 1-\hat\theta_{0|0}\]
According fomula (1) below,
\begin{equation}
\begin{pmatrix}\hat\alpha_1\\\hat\alpha_0\end{pmatrix}=\begin{pmatrix}\hat\theta_{1|1}+\hat\theta_{0|0}-1 & 0 \\ 0 & \hat\theta_{1|1}+\hat\theta_{0|0}\end{pmatrix}^{-1}\begin{pmatrix}
\frac{n_{11}}{n_{.1}}-1+\hat\theta_{0|0} \\
\frac{n_{10}}{n_{.0}}-1+\hat\theta_{0|0}
\end{pmatrix}
\end{equation}
using simulation data to get
\[\begin{array}{rcl}
\begin{pmatrix}\hat\alpha_1\\\hat\alpha_0\end{pmatrix}=\begin{pmatrix}0.939-0.761 & 0 \\ 0 & 0.939-0.761\end{pmatrix}^{-1}\begin{pmatrix}0.830-0.761 \\0.848-0.761\end{pmatrix}=\begin{pmatrix}0.388 \\0.486\end{pmatrix}
\end{array}\]
Hence, $\hat\alpha_1-\hat\alpha_0=-0.098$.

\item Find se($\hat\alpha_1-\hat\alpha_0$) by bootstrap($B=100$).\\
The bootstrap($B=100$) procedure is as follow,
\begin{lstlisting}
> WY <- cbind(W, Y)
> XWv <- cbind(Wv, Xv)
> alpdiff.boot <- NULL
> for(i in 1:100){
+   WY.boot <- WY[sample(500, replace = TRUE), ]
+   WY.tab.boot <- table(WY.boot[, 1], WY.boot[, 2])
+   n1.boot <- WY.tab.boot[2, 2]/sum(WY.tab.boot[, 2])
+   n0.boot <- WY.tab.boot[2, 1]/sum(WY.tab.boot[, 1])
+   XWv.boot <- XWv[sample(100, replace = TRUE), ]
+   XWv.tab.boot <- table(XWv.boot[, 1], XWv.boot[, 2])
+   theta1.boot <- XWv.tab.boot[2, 2]/sum(XWv.tab.boot[, 2])
+   theta0.boot <- XWv.tab.boot[2, 1]/sum(XWv.tab.boot[, 1])
+   theta.boot <- theta1.boot-theta0.boot
+   a.boot <- matrix(c(theta.boot, 0, 0, theta.boot), ncol=2) 
+   b.boot <- c(n1.boot, n0.boot)-theta0.boot
+   c.boot <- solve(a.boot)%*%b.boot
+   alpdiff.boot <- c(alpdiff.boot, c.boot[1]-c.boot[2])
+ }
> c(mean(alpdiff.boot), sd(alpdiff.boot))
[1] -0.1039442  0.2571436
\end{lstlisting}
Hence, the mean of bootstrap of ($\hat\alpha_1-\hat\alpha_0$)$=-0.104$ and se($\hat\alpha_1-\hat\alpha_0$)$=0.257$.

\end{enumerate}
%------------------------------------------------------------------
\item Choose 100 internal validation set of $(X, W, Y)$. Find estimates of $P(X=0, Y=0), P(X=1, Y=0),P(X=0, Y=1), P(X=1, Y=1)$ and compare with true value.\\
The internal validation set is randomly choose by the data in (a) and (b). The tables of internal validation set (\texttt{XWYu.tab}) and whole data (\texttt{WY.tab}) is as follow.
\begin{lstlisting}
> index <- sample(500, 100,replace = FALSE)
> XWYu <- cbind(X, W, Y)[index, ]
> (WY.tab <- table(W, Y))
   Y
W     0   1
  0  42  38
  1 234 186
> WY.joint <- matrix(WY.tab/sum(WY.tab))
> (XWYu.tab <- table(XWYu[, 1], XWYu[, 2], XWYu[, 3]))
X, W, Y = 0 
     0  1
  0  4 16
  1  4 33

X, W, Y = 1  
     0  1
  0  5 17
  1  1 20
\end{lstlisting}
According fomula (2) below,
\begin{multline}
\begin{pmatrix}\hat P(X=0, Y=0)\\\hat P(X=1, Y=0)\\\hat P(X=0, Y=1)\\\hat P(X=1, Y=1)\end{pmatrix}=\\
\begin{pmatrix}
P(X=0|W=0, Y=0) & P(X=0|W=1, Y=0) & 0 & 0\\
P(X=1|W=0, Y=0) & P(X=1|W=1, Y=0) & 0 & 0\\
0 & 0 & P(X=0|W=0, Y=1) & P(X=0|W=1, Y=1)\\
0 & 0 & P(X=1|W=0, Y=1) & P(X=1|W=1, Y=1)\\
\end{pmatrix}\\
\times\begin{pmatrix}\hat P(W=0, Y=0)\\\hat P(W=1, Y=0)\\\hat P(W=0, Y=1)\\\hat P(W=1, Y=1)\end{pmatrix}
\end{multline}
using simulation data to get
\[\begin{pmatrix}\hat P(X=0, Y=0)\\\hat P(X=1, Y=0)\\\hat P(X=0, Y=1)\\\hat P(X=1, Y=1)\end{pmatrix}=
\begin{pmatrix}
0.5 & 0.327 & 0 & 0\\
0.5 & 0.673 & 0 & 0\\
0 & 0 & 0.833 & 0.459\\
0 & 0 & 0.167 & 0.541\\
\end{pmatrix}
\begin{pmatrix}0.084\\0.468\\0.076\\0.372\end{pmatrix}=
\begin{pmatrix}0.194\\0.357\\0.234\\0.213\end{pmatrix}\]
Compare to question 1., \[\begin{pmatrix} P(X=0, Y=0)\\ P(X=1, Y=0)\\ P(X=0, Y=1)\\ P(X=1, Y=1)\end{pmatrix}=\begin{pmatrix} 0.200\\0.350\\0.300\\0.150\end{pmatrix}\]

%------------------------------------------------------------------

\end{enumerate}
\end{document}

