% These two lines must be incldued to open file under UTF-8
% !TEX TS-program = xelatex								
% !TEX encoding = UTF-8
% 套件設定
%\documentclass[11pt, a4paper, fleqn, titlepage]
% 			  {article}			            % 開起是否首頁置中
\documentclass[11pt, a4paper, fleqn]
 			  {article} 						
\usepackage{fontspec} 						% Font selection for XeLaTeX 
\usepackage{fourier}						% Adobe Utopia
\usepackage{xeCJK}							% 中文使用 XeCJK.
\usepackage{xunicode} 						% Unicode support for LaTeX character names.
\usepackage{amsmath, amssymb, amsthm, bm}	% math­e­mat­i­cal fea­tures found in AMS-TEX.
\usepackage{nccmath}						% cnter equation.
\usepackage{cases}				% Lists.
\usepackage{graphicx, subfig, float} 		% Support the \includegraphics command.
\usepackage{array, booktabs, threeparttable}% An ex­tended tab­u­lar en­vi­ron­ments.
\usepackage{indentfirst}					% 首行縮排
\usepackage{enumerate}
\usepackage[shortlabels]{enumitem}
\setlist{nosep}
\parindent=2em
\usepackage{xcolor}
\usepackage{listings}						% 映出程式碼
\usepackage[left=1.5cm, right=1.5cm, 
			bottom=2cm, top=25mm, 
			marginparwidth=0pt, 
			headheight=25mm]{geometry}		% cus­tomize page lay­out.
%\usepackage[left=1.5in,right=1in,
%			top=1in,bottom=1in]{geometry} 
\usepackage{setspace}						% 指定行距
\usepackage{fancyhdr}						% 套用頁首頁尾
%\usepackage[bibstyle=numeric]{biblatex}		% 參考書目

%\usepackage[style=reading,
%			bibstyle=numeric]{biblatex}		% 參考書目(社會學)\parencite{}
%\usepackage{dcolumn}
%\usepackage{longtable}
%\usepackage{colortbl}
%\usepackage{wrapfig, subfig, sidecap}		% 圍繞圖片,並排圖片,圖名放置。
%\usepackage{multicol}						% 多欄
\usepackage{tasks}
\settasks{label=(\alph*) ,label-width=1.5em,
          after-item-skip=-0.5em
          }
%\usepackage{icomma}						% 在數學式中,逗點後空格。
%--------------------------------------------------------------------------------------------------------------------------------------------
% 主字型設定
\setCJKmainfont{cwTeX Q Ming Medium}		% 設定中文內文字型
				[BoldFont=cwTeX Q Hei Bold]	% 設定中文粗體字型
\setmainfont{Times New Roman}				% 設定英文內文字型
\setsansfont{Arial}							% 設定英文無襯袖字型 used with {\sffamily ...}
\setmonofont{Courier New}					% 設定英文等寬度字型 used with {\ttfamily ...}


%--------------------------------------------------------------------------------------------------------------------------------------------
% 其他字型設定
\newfontfamily{\C}{Cambria}				
\newfontfamily{\A}{Arial}
\newfontfamily{\SCN}[Scale=0.9]{Courier New}
\newfontfamily{\TT}[Scale=0.8]{Times New Roman}
%\newCJKfontfamily{\MB}{微軟正黑體}
%\newCJKfontfamily{\SM}[Scale=0.8]{新細明體}		% 縮小版新細明體
%\newCJKfontfamily{\K}{標楷體}
\newCJKfontfamily{\CF}{cwTeX Q Fangsong Medium}	% CwTex 仿宋體
\newCJKfontfamily{\CB}{cwTeX Q Hei Bold}		% CwTex 粗黑體
\newCJKfontfamily{\CK}{cwTeX Q Kai Medium}   	% CwTex 楷體
\newCJKfontfamily{\CM}{cwTeX Q Ming Medium}		% CwTex 明體
\newCJKfontfamily{\CY}{cwTeX Q Yuan Medium}		% CwTex 圓體

%--------------------------------------------------------------------------------------------------------------------------------------------
\XeTeXlinebreaklocale "zh"					%這兩行一定要加,中文才能自動換行
\XeTeXlinebreakskip = 0pt plus 1pt			%這兩行一定要加,中文才能自動換行

%--------------------------------------------------------------------------------------------------------------------------------------------
\newcommand{\fib}[1]{\left(#1\right)}		% 設定 ()
\newcommand{\seb}[1]{\left[#1\right]}		% 設定 []
\newcommand{\thb}[1]{\left\{#1\right\}}		% 設定 []
\newcommand{\bsq}{\hfill$\blacksquare$}    	% 設定 \blacksquare
\newcommand{\pr}{\text{P\,}}    				% 設定 P
\newcommand{\Exp}{\text{E\,}}    				% 設定 E
\newcommand{\var}{\text{Var\,}}    			% 設定 Var
\newcommand{\cov}{\text{Cov\,}}    			% 設定 Var
\newcommand{\Oimgdir}{../image/}			% 設定圖檔的位置
\newcommand{\imgdir}{Code/}			    % 設定圖檔的位置
\renewcommand{\tablename}{表}				% 改變表格標號文字為中文的「表」(預設為 Table)
\renewcommand{\figurename}{圖}				% 改變圖片標號文字為中文的「圖」(預設為 Figure)
%--------------------------------------------------------------------------------------------------------------------------------------------
% 設定顏色 color Table: http://latexcolor.com
\definecolor{slight}{gray}{0.9}						

%--------------------------------------------------------------------------------------------------------------------------------------------
% 映出程式碼 \begin{lstlisting}
\lstset
{	language=R,  %[LaTeX]TeX
    breaklines=true,
    basicstyle=\linespread{0.5}\SCN,
    keywordstyle=\color{black}\bfseries,
    identifierstyle=\color{black},
%   commentstyle=\color{limegreen}\itshape,
%   stringstyle=\sffamily,
    showstringspaces=false,
    frame=single,							%default frame=none 
    rulecolor=\color{black},
    framerule=0.2pt,						%expand outward 
    framesep=3pt,							%expand outward
    xleftmargin=0em,						%to make the frame fits in the text area. 
    xrightmargin=0em,						%to make the frame fits in the text area. 
    tabsize=4,								%default :8 only influence the lstlisting and lstinline.
    escapeinside=`'                         %to cancel listing in part
}
