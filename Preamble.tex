% !TEX TS-program = xelatex								
% !TEX encoding = UTF-8
% 套件設定
\documentclass[12pt, a4paper, fleqn]{article} 		% 預設值為 10pt.
%\usepackage[left=1.5cm, right=1.5cm, 
%			bottom=2cm, top=25mm, 
%			marginparwidth=0pt, 
%			headheight=25mm]{geometry}		
\usepackage[left=3.18cm, right=3.18cm, 		% 版面大小(Socail Science).
			bottom=2.54cm, top=2.54cm, 
			marginparwidth=0pt, 
			headheight=2.54cm]{geometry}
%\usepackage{multicol}						% 多欄
\usepackage{titlesec}						% 自訂標題
\usepackage{zhnumber}						% 中文編號
\usepackage{tocloft}						% 修正目錄標號
\usepackage{fancyhdr}						% 套用頁首頁尾		
\usepackage{fontspec} 						% XeLaTeX 的字型選擇
\usepackage{fourier}						% Adobe Utopia 數學風格
\usepackage{xeCJK}							% 中文使用 XeCJK.
\usepackage{indentfirst}					% 首行縮排
%\usepackage{xcolor}							% 顏色套件
\usepackage{setspace}						% 指定行距
\usepackage{xunicode} 						% Unicode support for LaTeX character names.
\usepackage{amsmath, amssymb, amsthm}		% math­e­mat­i­cal fea­tures found in AMS-TEX.
\usepackage{enumerate, enumitem}			% Lists.
\usepackage{graphicx, subfig, float} 		% 圖片套件.
\usepackage{wrapfig, subfig, sidecap}		% 圍繞圖片,並排圖片,圖名放置。
\usepackage{array, booktabs, threeparttable}% 表格套件
\usepackage{longtable, dcolumn}				% 長表格、表格內小數點對齊
%\usepackage{listings, minted}				% 映出程式碼
\usepackage{listings}	        			% 映出程式碼
%\usepackage[bibstyle=numeric]{biblatex}	% 參考書目
\usepackage[style=apa]{biblatex}			% 參考書目(Socail Science)
\usepackage{multirow}                       % 表格工具
\usepackage[table,xcdraw]{xcolor}           % 表格工具
\usepackage{pdfpages}
\usepackage{bm}								% 數學粗體

%--------------------------------------------------------------------------------------------------------------------------------------------
% 主字型設定
\setCJKmainfont{標楷體}						% 中文內文字型
\setmainfont{Times New Roman}				% 英文內文字型
\setsansfont{Arial}							% 英文無襯袖字型  {\sffamily ...}
\setmonofont{Courier New}					% 英文等寬度字型  {\ttfamily ...}

%--------------------------------------------------------------------------------------------------------------------------------------------
% 其他字型設定
\newfontfamily{\A}{Arial}
\newfontfamily{\C}{Cambria}				
\newfontfamily{\SC}[Scale=0.9]{Cambria}
\newfontfamily{\TN}{Times New Roman}
\newCJKfontfamily{\K}{標楷體}
\newCJKfontfamily{\MB}{微軟正黑體}
\newCJKfontfamily{\NM}[Scale=0.9]{新細明體}			% 縮小版新細明體
\newCJKfontfamily{\BNM}{王漢宗粗明體繁}				% 粗體新細明體
%\newCJKfontfamily{\CwF}{cwTeX Q Fangsong Medium}	% CwTex 仿宋體
%\newCJKfontfamily{\CwH}{cwTeX Q Hei Bold}			% CwTex 粗黑體
%\newCJKfontfamily{\CwK}{cwTeX Q Kai Medium}   		% CwTex 楷體
%\newCJKfontfamily{\CwM}{cwTeX Q Ming Medium}		% CwTex 明體


%--------------------------------------------------------------------------------------------------------------------------------------------
\XeTeXlinebreaklocale "zh"					% 中文才能自動換行
\XeTeXlinebreakskip = 0pt plus 1pt			% 中文才能自動換行
\parindent=2em								% 首行空兩個字
\setlist{nosep}								% 阻止 list 自動間隔
%--------------------------------------------------------------------------------------------------------------------------------------------
\newcommand{\fib}[1]{\left(#1\right)}		% 設定 ()
\newcommand{\seb}[1]{\left[#1\right]}		% 設定 []
\newcommand{\thb}[1]{\left\{#1\right\}}		% 設定 []
\newcommand{\bsq}{\hfill$\blacksquare$}    	% 設定 \blacksquare
\newcommand{\pr}{\text{P\,}}    			% 設定 P
\newcommand{\Exp}{\text{E\,}}    			% 設定 E
\newcommand{\var}{\text{Var\,}}    			% 設定 Var
\newcommand{\cov}{\text{Cov\,}}    			% 設定 Var
%--------------------------------------------------------------------------------------------------------------------------------------------
\renewcommand{\tablename}{表}						% 改變表格標號文字為中文的「表」(預設為 Table)
\renewcommand{\figurename}{圖}						% 改變圖片標號文字為中文的「圖」(預設為 Figure)
\newcommand{\loflabel}{圖} 							% 圖目錄出現 圖 x.x 的「圖」字
\newcommand{\lotlabel}{表}  							% 表目錄出現 表 x.x 的「表」字

\renewcommand{\contentsname}{\hfill{\fontsize{16}{0}\selectfont 目次}\hfill}  
\renewcommand{\cftaftertoctitle}{\hfill}
\renewcommand{\listtablename}{\hfill{\fontsize{16}{0}\selectfont 表次}\hfill} 
\setlength{\cftfigindent}{0pt} 
\renewcommand{\listfigurename}{\hfill{\fontsize{16}{0}\selectfont 圖次}\hfill}  
\setlength{\cfttabindent}{0pt}

%--------------------------------------------------------------------------------------------------------------------------------------------
\setcounter{tocdepth}{2}							% 設定目錄只到subsection
\renewcommand\thesection{第\zhnum{section}章}		% 設定section標號中文
\renewcommand\thesubsection{第\zhnum{subsection}節}
\renewcommand\thesubsubsection{\zhnum{subsubsection}、}

\titleformat{\section}[hang]{\centering\fontsize{18}{0}\selectfont}		% 設定 section 標號、標題字體樣式
{\thesection}{1em}{}													% 設定 section 標號與標題間格
\titlespacing{\section}{0pt}{*10}{*5}									% 設定 section 段落間格
\titleformat{\subsection}[hang]{\centering\fontsize{16}{0}\selectfont}
{\thesubsection}{1em}{}
\titlespacing{\subsection}{0pt}{*10}{*3}	
\titleformat{\subsubsection}[hang]{}
{\thesubsubsection}{0.1cm}{}


\cftsetindents{section}{0em}{4em}					% 設定 section 目錄間格{標號前空格數}{標號前+標題前空格數}
\cftsetindents{subsection}{1em}{4em}

%\addtolength\cftsecnumwidth{2em}
%\addtolength\cftsubsecnumwidth{2em}
%\addtolength\cftsubsubsecindent{1em}



%--------------------------------------------------------------------------------------------------------------------------------------------
% 設定顏色 color Table: http://latexcolor.com
\definecolor{slight}{gray}{0.9}						

%--------------------------------------------------------------------------------------------------------------------------------------------
% 映出程式碼 \begin{lstlisting}
\lstset
{	language=R,  %[LaTeX]TeX
    breaklines=true,
    basicstyle=\linespread{0.9}\tt,
    keywordstyle=\color{black}\bfseries,
    identifierstyle=\color{black},
%   commentstyle=\color{limegreen}\itshape,
    stringstyle=\sffamily,
    showstringspaces=false,
    frame=lines,							%default frame=none 
    rulecolor=\color{black},
    framerule=0.2pt,						%expand outward 
    framesep=3pt,							%expand outward
    xleftmargin=2em,						%to make the frame fits in the text area. 
    xrightmargin=3pt,						%to make the frame fits in the text area. 
    tabsize=4,								%default :8 only influence the lstlisting and lstinline.
    escapeinside=`'                         %to cancel listing in part
}

%\usemintedstyle{borland}
%\lstset{language=SAS, 
%  breaklines=true,  
%  basicstyle=\ttfamily\bfseries,
%  columns=fixed,
%  keepspaces=true,
%  identifierstyle=\color{blue}\ttfamily,
%  keywordstyle=\color{cyan}\ttfamily,
%  stringstyle=\color{purple}\ttfamily,
%  commentstyle=\color{green}\ttfamily,
%  } 